\documentclass[12pt, USenglish]{article}  % Science: US variant of English.

\usepackage{babel}
\usepackage[babel=true]{microtype}
\usepackage[breaklinks, pdfborder={0 0 0}]{hyperref}
\usepackage{csquotes}
\usepackage{amsmath}
\usepackage{units}
\usepackage{tikz}
\usepackage{sansmath}  % Science: sans serif in math in figures.
% Science: Caption labels are boldface "Fig. X." and "Table X.".
\usepackage{caption}
\usepackage{isodate}
\captionsetup{labelfont=bf, labelsep=period, figurename={Fig.}}
\addto\extrasUSenglish{\renewcommand{\figureautorefname}{Fig.}}

% Setup figure numbering.
\setcounter{figure}{0}

% Science: Prepend 'S' to the appendix number.
\newcommand{\appendixprefix}{S}
\renewcommand{\thefigure}{\appendixprefix\arabic{figure}}


\pagestyle{empty}


\begin{document}

\section*{Supplementary figures}

\begin{figure}[h!]
  \centering
  \includegraphics{../population_size}
  \caption{The sensitivity of extinction time of FMDV to buffalo
    population size.
    For each model and each SAT, the model was simulated for 1000 runs
    at population sizes 100, 200, 300, 400, 500, 600, 700, 800, 900,
    1000 (baseline, dotted vertical lines), 2000, 3000, 4000, and
    5000.
    The other parameters were fixed at their baseline values.
    The top and middle rows of graphs show the distribution of
    FMDV extinction times for the model with only acute transmission
    and the model with both acute and carrier transmission,
    respectively.
    The bottom row shows the proportion of simulations where FMDV
    persisted in the buffalo population for the whole simulated
    10-year period for the model with both acute and carrier
    transmission.}
\end{figure}


\begin{figure}
  \centering
  \includegraphics{../birth_seasonality}
  \caption{The sensitivity of extinction time to birth seasonality.
    For each model and each SAT, the model was simulated for
    1000 runs at 0, 0.5, 1 (baseline, dotted vertical lines), 1.5, and
    2 times the baseline birth seasonal coefficient of variation of
    0.613.
    The other parameters were fixed at their baseline values.
    The top and middle rows of graphs show the distribution of
    FMDV extinction times for the model with only acute transmission
    and the model with both acute and carrier transmission,
    respectively.
    The bottom row shows the proportion of simulations where FMDV
    persisted in the buffalo population for the whole simulated
    10-year period with both acute and carrier transmission.}
\end{figure}


\begin{figure}
  \centering
  \includegraphics{../samples_sensitivity_acute}
  \caption{The sensitivity of FMDV extinction time to model
    parameters, for the model with only acute transmission.
    The sensitivity is measured by the partial rank correlation
    coefficient (PRCC). The model was simulated with each of 20,000
    samples from the posterior distributions of the parameters
    (Fig.~2, Table S1).}
\end{figure}

\begin{figure}
  \centering
  \includegraphics[width=\textwidth]{../FigS4_Experimental_Design_crop}
  \caption{Experimental design: (A) FMDV transmission during acute infection. In separate corrals for each serotype (SAT1, SAT2, SAT3), four needle-infected animals were mixed with four naïve buffalo, two days after the initial infectious challenges (needle infections). Presence of virus and antibodies to FMDVs were tested in the recipient animals on days 2, 4, 6, 8, 11, 14 and 30 post exposure to the needle infected buffalo. (B) Transmission from FMDV carriers. FMDV carrier status is defined as retention of FMDV for more than 30 days after primary infection. In two separate corrals, two groups of six carrier animals, including two per serotype, were mixed with two groups naïve animals at day 45 post-infection. Buffalo were sampled to test for FMDV transmission after two weeks post-contact, and then monthly for six months.}
\end{figure}

\begin{figure}
  \centering
  \includegraphics[width=\textwidth]{../sup_figure5.pdf}
  \caption{FMDV transmission events from carrier to naïve
    buffalo. Carrier buffalo are shown in blue (SAT 1), red (SAT 2) and purple (SAT 3). Lighter shades indicate loss of carrier status in these individuals. Yellow shading demarks new infections that occurred during the experimental period. The numbers are the log10 titre of virus genome per \unit{µl} of sample using specific
    primers for each serotype (methods and results described in
    \textit{16}). Samples categorized as positive (pos) were RT PCR
    positive but the Ct value was above a cut off of 32
    (\textit{16}). Carrier status was confirmed on day 30
    post-infection. Experimental groups, each including two carriers
    of each serotype and six naïve buffalo, were assembled on day
    46. All buffalo were sampled for FMDV testing 2 weeks
    post-contact, and subsequently monthly for 6 months. Samples from
    day 120 and day 210 have not been analyzed and are not shown
    here. By two weeks post-contact, at least one transmission event
    from a SAT1 carrier and one from a SAT3 carrier had occurred in
    group one, and by 6 weeks post-contact at least one more
    transmission event from a SAT1 carrier had occurred in group
    2. Additional transmission events may have originated from
    carriers or may reflect secondary infections within each group.}
\end{figure}

\begin{figure}
  \centering
  \textsf{\begin{tikzpicture}[compartment/.style={rectangle, draw},
                    font=\fontsize{5pt}{6}\selectfont]
  % Compartments.
  \node at (0, 8.5) [compartment, align=center, name=MaternalImmunity] {Maternal\\immunity};
  \node at (0, 6.5) [compartment, name=Susceptible] {Susceptible};
  \node at (0, 4.5) [compartment, name=Exposed] {Exposed};
  \node at (0, 2.5) [compartment, name=Infectious] {Infectious};
  \node at (0, 0) [compartment, name=Recovered] {Recovered};
  \node at (2.5, 0.625) [compartment, dashed, name=Chronic] {Chronic};

  % Location for branch from Infectious to Chronic and Recovered.
  \coordinate (recovery) at (0, 1.25);

  % Infection-related processes.
  \draw [->] (MaternalImmunity)
             to node [rotate=90, above] {waning}
             (Susceptible);
  \draw [->] (Susceptible)
             to node [rotate=90, above] {infection}
             (Exposed);
  \draw [->] (Exposed)
             to node [rotate=90, above] {progression}
             (Infectious);
  \draw [  ] (Infectious)
             to node [rotate=90, above, yshift=-1pt] {recovery}
             (recovery);
  \draw [->, dashed] (recovery)
             to node [sloped, above, yshift=-2pt] {chronicity}
             (Chronic.161);
  \draw [->] (recovery)
             to node [] {}
             (Recovered.90);
  \draw [->, dashed] (Chronic.199)
             to node [sloped, align=center] {chronic\\recovery}
             (Recovered.15);
  % \draw [->] (Recovered.195)
  %            to [out=180, in=180] node [left, align=center] {immunity\\waning}
  %            (Susceptible.180);

  % Births
  \draw [->] (Susceptible.196)
             to [out=225, in=180, looseness=3.5] node [] {}
             (Susceptible.180);
  \draw [->] (Exposed.180)
             to [out=180, in=180] node [] {}
             (Susceptible.180);
  \draw [->] (Infectious.180)
             to [out=180, in=180, looseness=0.9] node [sloped, above, pos=0.85] {birth}
             (Susceptible.180);
  \draw [->] (Recovered.180)
             to [out=180, in=180, looseness=0.6] node [sloped, above, pos=0.8] {birth}
             (MaternalImmunity.180);
  \draw [->, dashed] (Chronic.90)
             to [out=90, in=0, looseness=0.65] node [sloped, above, pos=0.75] {birth}
             (MaternalImmunity.0);

  % Deaths
  \draw [->] (MaternalImmunity.334)
             to node [sloped, below, yshift=1pt] {death}
             +(315: 1);
  \draw [->] (Susceptible.344)
             to node [sloped, below, yshift=1pt] {death}
             +(315: 1);
  \draw [->] (Exposed.340)
             to node [sloped, below, yshift=1pt] {death}
             +(315: 1);
  \draw [->] (Infectious.345)
             to node [sloped, below, yshift=1pt] {death}
             +(315: 1);
  \draw [->] (Recovered.345)
             to node [sloped, below, yshift=1pt] {death}
             +(315: 1);
  \draw [->, dashed] (Chronic.342)
             to node [sloped, below, yshift=1pt] {death}
             +(315: 1);
\end{tikzpicture}

%%% Local Variables:
%%% mode: latex
%%% TeX-master: "diagram_standalone"
%%% End:
}
  \caption{Model diagram. The dashed box and arrows show the
   state and transitions present in the model with both acute and
   carrier transmission that are not present in the model with only
    acute transmission.}
\end{figure}

\begin{figure}
  \centering
  \begin{sansmath}
    \input{birth_hazard.pgf}
  \end{sansmath}
  \caption{Model birth hazards for ages $\unit[4]{y}$ and older.}
\end{figure}

\begin{figure}
  \centering
  \begin{sansmath}
    \input{distributions.pgf}
  \end{sansmath}
  \caption{Hazards and survivals for the model events.}
\end{figure}

\begin{figure}
  \centering
  \begin{sansmath}
    \input{stable_age_distribution.pgf}
  \end{sansmath}
  \caption{The stable age distribution of the buffalo population
    on {\printyearoff\printdate{2020-07-16}}, $t_0 = \unit[0.5]{y}$
    after the peak in the birth hazard.}
\end{figure}

\begin{figure}
  \centering
  \includegraphics{../start_time}
  \caption{The sensitivity of extinction time to model start time.
    For each model and each SAT, the model was simulated for
    1000 runs starting $t_0 = \unit[i / 12]{y}$ after the peak in the
    birth hazard, for $i = 0, 1, \dots, 11$. The baseline value is
    $t_0 = \unit[0.5]{y}$ (dotted vertical lines). The other
    parameters were fixed at their baseline values. The top and middle
    rows of graphs show the distribution of FMDV extinction times for
    the model with only acute transmission and the model with both
    acute and carrier transmission, respectively. The bottom row shows
    the proportion of simulations where FMDV persisted in the buffalo
    population for the whole simulated 10-year period with both acute
    and carrier transmission.}
\end{figure}

\begin{figure}
  \centering
  \textsf{\begin{tikzpicture}[compartment/.style={rectangle, draw},
                    font=\fontsize{5pt}{6}\selectfont]
  % Compartments.
  \node at (0, 4.5) [compartment, align=center, name=MaternalImmunity] {Maternal\\immunity};
  \node at (0, 2.5) [compartment, name=Susceptible] {Susceptible};
  \node at (0, 0) [compartment, name=Recovered] {Recovered};
  \node at (2.5, 0.625) [compartment, dashed, name=Carrier] {Carrier};

  % Location for branch from Susceptible to Carrier and Recovered.
  \coordinate (recovery) at (0, 1.25);

  % Infection-related processes.
  \draw [->] (MaternalImmunity)
             to node [rotate=90, above] {waning}
             (Susceptible);
  \draw [->] (Susceptible)
             to node [rotate=90, above, yshift=-1pt] {infection}
             (recovery);
  \draw [->, dashed] (recovery)
             to node [sloped, align=center] {probability\\carrier}
             (Carrier.159);
  \draw [->] (recovery)
             to node [] {}
             (Recovered.90);
  \draw [->, dashed] (Carrier.200)
             to node [sloped, align=center] {carrier\\recovery}
             (Recovered.15);
\end{tikzpicture}
}
  \caption{Diagram of the simplified model used in finding initial
    conditions for the full model. The dashed box and arrows show the
    state and transitions present in the model with both acute and
    carrier transmission that are not present in the model with only
    acute transmission.}
\end{figure}

\begin{figure}
  \centering
  \includegraphics{../initial_conditions_acute}
  \caption{The sensitivity of extinction time to birth seasonality for
    the model with acute transmission only.
    For each SAT (rows of paired graphs of number infected and
    extinction time), the model was simulated for 1000 runs using the
    baseline initial conditions for each SAT (columns).
    The other parameters were fixed at their baseline values.
    E.g.~the model with the baseline parameter values for SAT 1 (top
    row) was simulated with the baseline initial conditions for SATs
    1, 2, and 3.
    In the graphs of number infected, the thin colored curves show the
    number infected vs. time for the individual simulations, while the
    thick black curve is the mean over the simulations of the number
    infected vs. time.
    The graphs of FMDV extinction time show the distribution of
    extinction times (when the number infected first becomes 0) over
    the simulations.}
  \label{fig:initial_conditions_acute}
\end{figure}

\begin{figure}
  \centering
  \includegraphics{../initial_conditions_chronic}
  \caption{The sensitivity of extinction time to birth seasonality for
    the model with transmission from acutely infected and carrier hosts.
    For each SAT (rows of paired graphs of number infected and
    extinction time), the model was simulated for 1000 runs using the
    baseline initial conditions for each SAT (columns).
    The other parameters were fixed at their baseline values.
    E.g.~the model with the baseline parameter values for SAT 1 (top
    row) was simulated with the baseline initial conditions for SATs
    1, 2, and 3.
    In the graphs of number infected, the thin colored curves show the
    number infected vs. time for the individual simulations, while the
    thick black curve is the mean over the simulations of the number
    infected vs. time.
    The graphs of FMDV extinction time show the distribution of
    extinction times (when the number infected first becomes 0) over
    the simulations, arrows show the
    proportion of simulations that persisted longer than 10 years,
    and the gray boxes show the longest persistence time for the
    model with acute transmission only to highlight the difference in
    scale.}
\end{figure}


\end{document}
