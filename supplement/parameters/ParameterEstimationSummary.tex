\documentclass[letterpaper,12pt]{article}
\usepackage{booktabs,graphicx,amssymb,lineno,amsmath,multirow,rotating,verbatim,setspace,float,subfigure, fixltx2e}
\usepackage[margin=1in]{geometry}
\usepackage{gensymb}
\usepackage[table]{xcolor}
\renewcommand{\vec}[1]{\mathbf{#1}}
\newcommand{\mat}[1]{\mathbf{#1}}
\DeclareMathOperator{\Prob}{Prob}
\newcommand{\md}{\mathrm{d}}
\newcommand{\me}{\mathrm{e}}
\newcommand{\mT}{\mathrm{T}}
\allowdisplaybreaks[1]
\usepackage{units}

\begin{document}

%\linenumbers

\renewcommand{\thesubfigure}{\Alph{subfigure}}

%\bibpunct{[}{]}{,}{n}{,}{,}
%\onehalfspace
\title{Methods 2: Parameter Summary}
\author{Erin Gorsich}
\date{18-May-2016}

\maketitle
\doublespacing

\section{Survival rates}
We took the annual survival to be
\begin{equation}
  \Prob\{\text{Survival for $\unit[1]{yr}$}\}
  =
  \begin{cases}
    0.66 & \text{if $a < 1$},
    \\
    0.79 & \text{if $1 \leq a < 3$},
    \\
   0.88 & \text{if $3 \leq a < 12$},
    \\
    0.66 & \text{if $a \geq 12$}.
  \end{cases}
\end{equation}

Published survival or mortality rates were variable (Figure 4).  
Mortality rates ranged from $<5\%$ to over $40\%$ depending on the age, sex, and disease assumptions of the study (Cross et al. 2009; Cross and Getz 2006; Jolles et al. 2005; Gorsich et al. \textit{in prep}).  
In the most recent cohort study of African buffalo in the Lower Sabie and Crocodile Bridge region of Kruger National Park (KNP), overall adult and juvenile survival rates were estimated as 0.88 and 0.79.  
When survival rates were estimated for the two regions separately, adult survival was 0.92 and 0.86 and juvenile survival was 0.83 and 0.68 for the Lower Sabie and Crocodile Bridge region, respectively (Gorsich et al. \textit{in prep}).  
Therefore, we explore a range of survival rates that span this site specific variation as well the range of published adult and juvenile survival rates (Table 1).  
Survival rates in calves and geriatric buffalo were estimated as 0.76 times smaller than adult survival (Jolles 2005).
%These two parameters were variable between studies, so they are varied from 0.2 to 0.8. 

\begin{figure}[h]
\begin{center}
\includegraphics[width=5in]{Survival_parameters.pdf}
\caption{Published estimates of buffalo survival in South Africa (Cross and Getz 2006, Cross et al. 2009, Jolles et al. 2005, Funston and Mills, 2006}
\end{center}
\end{figure}

\section{Birth} 
Approximately 30\% of our buffalo aged $\geq 4$ captured from May-August (LS) had a calf. 
Approximately 55\% of our buffalo aged $\geq 4$ captured from October-January (LS) were pregnant.  
In the model, this is set to allow a constant population size. 
\pagebreak

\section{Waning} 
Currently, everyone looses maternal immunity 6 months after birth.  \\

\section{Infetion} 

Experimental infections of African buffalo with SAT-1, SAT-2, or SAT-3 were recently conducted (Perez et al. 201?).
In this study, 4 buffalo were infected with FMDV by (vet word for injections in the toung)..  
Two days post-infection, all experimental buffalo were viremic and 4 naive buffalo were introduced.
Viremia was monitored at two day intervals by real-time PCR from circulating blood samples
Additional data collection methods and analyses are reported in Perez et al. (201?)
Experimental infections were conducted separately for each SAT type. 
We assume animals are infected on the first sampling period they are viremic and recovered on sampling periods when they are no longer viremic.
The timing of viremia in the four, naive buffalo informs our estimate of the transmission rate. \\

We estimate the transmission rate of each SAT-type, $\beta$, by relating the assumptions in our SIR model to the probability that a contact buffalo becomes infected in each sampling period.
Specifically, we assume density dependent transmission where the rate of infectious contacts is proportional to the density of infectious animals (Begon et al. 2002).
Thus, the number of infectious contacts follows a Poisson distribution with mean, $\lambda_t = \beta  I_t$, where $I_t$ is the number of infectious animals at time. 
A statistical model to relate data on counts of animals becoming infected, $C_t$ to the SIR model was formalized by Becker et al. (1989) and reviewed in Velthuis et al. (2007). \\
% and its application to disease studies are common (Correia-Gomes et al. 2014; Thomas et al. 2011; Tsutsui et al. 2010)

The probability that a susceptible animal becomes infected over a sampling window, $\Delta t$, is $p_t = 1 - \text{exp}(- \lambda_t \Delta t)$.
In reality, $\lambda_t$ will vary across our sampling period as the number of infected animals is subject to recovery and transmission.  
We estimate the number of infected animals contributing to the force of infection as the average number in the sample period.
Then, the counts of animals becoming infected $C_t$ follow a binomial distribution (Becker et al. 1989), specified as, 

\begin{align}
C_t &\sim \text{Binomial} \Big( S_t \text{, }p_t\Big) \nonumber \\  %note use align* if no numbers at all
p_t &= 1 - \text{exp}\Big(-\beta (I_t + I_{t+1})\Delta t / 2 \Big)
\end{align}

\begin{figure}[h]
\begin{center}
\includegraphics[width=5in]{ML_TransmissionRate.pdf}
\caption{Likelihood profiles of the transmission rate for SAT-1, SAT-2, SAT-3 based on experimental infection data.  Note that for SAT-1, all contact animals became infected within the first two days, so we cannot estimate the transmission rate.  The lighter colored lines are the likelihood profiles if we assume the number of buffalo contributing to transmission can be approximated by the number infectious buffalo at the start of the sample period rather than the average.}
\end{center}
\end{figure}

\begin{figure}[h]
\begin{center}
\includegraphics[width=5in]{Transmission_parameters.pdf}
\caption{Estimated transmission rate.}
\end{center}
\end{figure}


\clearpage
\section{Recovery} 
We assume that the rate, $\gamma$, animals loose viremia is constant.
The probability an infected buffalo at time t, will no longer be viremic by the subsequent sampling period is, $p_R = 1- e^{- \gamma \Delta t}$, were $\gamma$ is the recovery rate and $\Delta t$ will reflect our two day sampling periods.  
The number of newly recovered animals, $R_{t}$, at each time step starting at time, t, can be represented with a binomial distribution, 
\begin{align}
R_{t} &\sim \text{Binomial}(I_{t}  , p_R). \nonumber \\
p_R &= 1- e^{- \gamma \Delta t}
\end{align}

Data on both experimentally and naturally infected buffalo are considered in this estimate, such that the initial number of infected buffalo is 8 and no new infections are considered.
Although sampling occurred in 2 day windows for 35 days after the experimental infection, no animals remained viremic for longer than 6 days.

Given the binomial probability mass function and the observed counts of recovered individuals, $R_{t}$, during $t=1, 2, 3$ sampling periods after the initial viremia was detected, the maximum likelihood estimate for the recovery rate is the value of $\gamma$ that maximizes $L(\gamma)$, defined by the following, 
\begin{align}
P(R_{t} = r | I_t, p_R) &= \binom{I_t}{r} (p_R)^{r} (1 - p_R)^{I_t-r}  \nonumber  \\
&=  \binom{I_t}{r} (1-e^{- 2 \gamma})^{r} (e^{- 2 \gamma})^{I_t-r} \nonumber \\
L(\gamma) &= \sum_{t=1}^{3} ln \Big(  \binom{I_t}{r} (1- e^{- 2 \gamma})^{r} (e^{- 2 \gamma})^{I_t-r}    \Big)
\end{align}

The maximum likelihood estimate of the recovery rate was 0.23 (95\% CI 0.10 - 0.42) for SAT-1,  0.20 (95\% CI 0.08 - 0.41) for SAT-2, and 0.38 (95\% CI 0.17 - 0.72) for SAT-3.  95\% Confidence intervals are estimated based on the profile likelihood method (Figures 4 and 5). \\

\begin{figure}[h]
\begin{center}
\includegraphics[width=5in]{ML_InfectionDuration.pdf}
\caption{Likelihood profiles of the infection duration for SAT-1, SAT-2, SAT-3 based on experimental infection data}
\end{center}
\end{figure}

\begin{figure}[h]
\begin{center}
\includegraphics[width=5in]{InfectionDuration_parameters.pdf}
\caption{Corresponding mean and 95\% Confidence Interval estimates for the duration of viremia for each sero-type}
\end{center}
\end{figure}

\clearpage



%\newpage
\clearpage


%\textbf{Interesting stuff} \\
%Tenzin et al. (2008) do a meta-analysis to estimate the transmission rate of FMDV from carriers.  They have a list of experimental infections, sample sizes for each study, SIR and N, for all 2 buffalo-to-buffalo studies and many buffalo to cattle studies.

\section{Parameter Table} 
\begin{table}[ht]
\scriptsize
\centering
\begin{tabular}{l c c c c c}
\hline\hline
Parameter & Estimate & Range & Definition  \\ [0.5ex]
\hline 
Calf Survival & 0.66 & 0.4- 0.95 & Annual survival rate for buffalo aged < 1 year (1/yr)\\ % say discrete/cont;unit bins 
Juvenile Survival & 0.79 & 0.4- 0.95 & Annual survival rate for buffalo aged 1-3 years (1/yr) \\
Adult Survival & 0.88 & 0.4- 0.95 &  Annual survival rate for buffalo aged 3-12 years (1/yr)\\
Geriatric Survival & 0.66 & 0.4- 0.95 & Annual survival rate for buffalo aged $\geq$ 12 years (1/yr)  \\
Maternal Immunity & 0.5 & 0.2- 0.8 & Duration that maternal antibodies are protective (1/yr)\\
Transmission rate ($\beta$), SAT-1 & $\textgreater$ 0.09 & NA & Transmission rate of SAT-1 (1/day)\\
Transmission rate ($\beta$), SAT-2 & 0.09 & 0.03- 0.18 & Transmission rate of SAT-2 (1/day) \\
Transmission rate ($\beta$), SAT-3 & 0.05 & 0.01- 0.13 & Transmission rate of SAT-3 (1/day) \\
Recovery rate ($\gamma$), SAT-1 & $\frac{1}{4.4}$ &  $\frac{1}{9.7}-  \frac{1}{2.4}$ & Inverse of the duration of viremia for SAT-1 (1/days) \\
Recovery rate ($\gamma$), SAT-2 & $\frac{1}{4.9}$ &  $\frac{1}{12.5}-  \frac{1}{2.4}$ & Inverse of the duration of viremia for SAT-2 (1/days) \\
Recovery rate ($\gamma$), SAT-3 & $\frac{1}{5.8}$ &  $\frac{1}{2.5}-  \frac{1}{1.4}$ & Inverse of the duration of viremia for SAT-3 (1/days)\\
\hline
\end{tabular}
\label{table:nonlin}
\end{table}
% Table B1: List of literature including the degree distribution information included in this study.  For each study, the method of data collection (B: Behavioral observations; G: Group observations; CMR: Capture-Mark-Recapture techniques; P: Proximity collar or PIT tags; S: Spatial observations), whether the network and degree distribution were weighted or unweighted, and how the study handed individuals with no contacts. 

In the model we specify $R_o = \frac{\beta N}{\gamma}$.  
$R_o$ values based on the mean transmission and recovery rates estimated above in the 8 experimental buffalo are $R_o = 3.5$ for SAT-2 and $R_o = 2.32$ for SAT-3.  
These values only hold under the assumptions of the model. 
Naively scaling the transmission rate captured in the experimental herd to the whole herd \textit{may} be inappropriate because we are assuming the rate of contacts in the boma, c, is proportional to the density of the population (i.e. $c = \frac{\kappa N}{A}$, where A is the area; Begon et al. 2002).
Therefore, we will think about how contacts scale with area by using Julie's work for the second round of figures.  

\section{Results using parameters above}
When specified as: \\
SAT1: $R_o = 3.5; \gamma= 1/4.4$, cv = 0.6 \\
SAT2: $R_o = 3.5, \gamma = 1/4.9$, cv = 0.6\\
SAT3: $R_o = 2.3, \gamma = 1/5.8$, cv = 0.6 \\


\begin{figure}[h]
\begin{center}
\includegraphics[width=5in]{modelrun_SAT2.pdf}
\caption{SAT- 2 dynamics- all infections burn out before a year}
\end{center}
\end{figure}

\begin{figure}[h]
\begin{center}
\includegraphics[width=5in]{modelrun_SAT3.pdf}
\caption{SAT- 3 dynamics- all infections burn out before a year}
\end{center}
\end{figure}

\begin{figure}[h]
\begin{center}
\includegraphics[width=5in]{modelrun_20daydurationRo4.pdf}
\caption{Dynamics with 20 day infection duration (4 times our estimate), same Ro = 4.  Note some make it over a year.}
\end{center}
\end{figure}

%\section{Citations used (not updated)} 
%Moonen, P., and Schrijver, R.(2000). Carriers of foot-and-mouthdisease virus: A review. Veterinary Quarterly, 22, 193?197.5. 
%Salt, J. S. (1994). The epidemiological significance of FMDVcarriers: A review. Society for Veterinary Epidemiology andPreventive Medicine, Proceedings,71?84 
%Tenzin et al. 2008. Rate of Foot-and-Mouth Disease Virus Transmission by Carriers Quantified from Experimental Data. Risk Analysis, 28: 303-309.
\end{document}
