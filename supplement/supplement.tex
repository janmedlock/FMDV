\documentclass[12pt]{article}

\usepackage[T1]{fontenc}
\usepackage{textcomp}

\usepackage[english]{babel}
\usepackage[utf8]{inputenc}
\usepackage{csquotes}

\usepackage{lmodern}

\usepackage{hyperref}
\hypersetup{breaklinks}
\hypersetup{pdfborder=0 0 0}

\usepackage[babel=true]{microtype}

\usepackage{amsmath}
\renewcommand{\vec}[1]{\mathbf{#1}}
\newcommand{\mat}[1]{\mathbf{#1}}
\DeclareMathOperator{\Prob}{Prob}
\newcommand{\md}{\mathrm{d}}
\newcommand{\me}{\mathrm{e}}
\usepackage{sansmath}  % To use sans serif in math in figures.

\usepackage{units}
\usepackage{tikz}

\usepackage[style=nature, autocite=superscript, sortcites=true]{biblatex}
\addbibresource{supplement.bib}

\usepackage[UKenglish]{isodate}
\cleanlookdateon

% Set up numbering.
\newcommand{\appendixlabel}{S}
\newcounter{chapter}
\setcounter{chapter}{3}
\renewcommand{\thechapter}{\appendixlabel\arabic{chapter}}
\renewcommand{\thesection}{\thechapter.\arabic{section}}
\renewcommand{\thefigure}{\thechapter.\arabic{figure}}
\renewcommand{\thetable}{\thechapter.\arabic{table}}
\renewcommand{\theequation}{\thechapter.\arabic{equation}}


\title{\emph{Endemic dynamics of foot-and-mouth disease viruses in
    their reservoir, African buffalo}\\
  Appendix \thechapter: Model development and analysis}

\author{Anna Jolles \and Erin Gorsich \and Simon Gubbins
  \and Brianna Beechler \and Peter Buss \and Nick Juleff
  \and Lin-Mari deKlerk-Lorist \and Francois Maree
  \and Eva Perez-Martin \and OL van Schalkwyk \and Katherine Scott
  \and Jan Medlock \and Bryan Charleston}


\begin{document}

\maketitle


We built a stochastic individual-based model to capture the dynamics
of FMDV in African buffalo.  The software we wrote to simulate and
analyze the model is available for free under an open-source license
\autocite{medlock_2020}. This software was written in the Python
programming language \autocite{python}.

In the model, the age and sex of each buffalo is tracked along with
its immune state (\autoref{fig:diagram}): either immune due to
maternal antibodies ($M$), susceptible to infection ($S$),
exposed ($E$), infectious ($I$), carrier ($C$), or recovered ($R$).
There are 7 events that can occur to each buffalo:
\begin{description}
\item[Death] On the birth of new buffalo calf, the age at death of
  that calf is sampled from the mortality distribution.

\item[Birth] For each female buffalo, the time until she gives birth
  to a calf is sampled from its distribution.  This is done when the
  female is herself born, to find the time until she gives birth to
  her first calf, and after a birth, to find the time until she gives
  birth to her next calf.  A simple Bernoulli sample determines the
  sex of each calf.

\item[Waning] Each new buffalo calf is assumed to be
  immune to infection due to maternal antibodies: at birth, the
  duration of maternal immunity is sampled from its distribution.

\item[Infection] For each susceptible buffalo, the time to infection
  is sampled from its distribution, which depends on the current
  number of infected buffalo in the population.

\item[Progression] On infection, the time to progression is sampled
  from its distribution.

\item[Recovery] On infection, the time to recovery is sampled from its
  distribution.  A simple Bernoulli sample determines whether the
  recovered buffalo becomes a carrier.

\item[Chronic recovery] When a buffalo becomes a carrier, the time to
  recovery is sampled from its distribution.
\end{description}
The distributions that govern these processes are detailed below.
(See also \autoref{fig:distributions}.) When sampling from standard
distributions, we used the SciPy library \autocite{scipy}.  For
non-standard distributions, we used the inverse transform method
\autocite{rubinstein_1981} for sampling.


\begin{figure}
  \centering
  \textsf{\input{../diagram/diagram.tex}}
  \caption{Model diagram.  The dashed box and arrows show the state
    and transitions present in the chronic model but not the acute
    model.}
  \label{fig:diagram}
\end{figure}


The model simulations follow a Gillespie algorithm
\autocite{gillespie_1977}. For each buffalo, a list of events and the
times they occur is stored. The next event over the whole population
is found and the population is updated.  The hazards for infection
depend on the number of infectious buffalo in the population and so
the times to infection are updated after each change in the
population.  The hazards of the other events are independent of the
state of the population and so the times to these events are not
updated.  This process was repeated from $t = 0$ to until there were
$0$ infected (exposed, infectious, and chronic) buffalo or until $t =
\unit[10]{y}$. The simulations were stopped at $t = \unit[10]{y}$ to
limit the total computation time of running many simulations.

In the following, the variables $t$ and $a$ denote time and age,
respectively.

\section{Death}

We took the annual survival to be
\begin{equation}
  \Prob\{\text{Survival for $\unit[1]{y}$}\}
  =
  \begin{cases}
    0.66 & \text{if $a < \unit[1]{y}$},
    \\
    0.79 & \text{if $\unit[1]{y} \leq a < \unit[3]{y}$},
    \\
    0.88 & \text{if $\unit[3]{y} \leq a < \unit[12]{y}$},
    \\
    0.66 & \text{if $a \geq \unit[12]{y}$}.
  \end{cases}
\end{equation}
Assuming that the mortality hazard is constant throughout a year gives
the hazard
\begin{equation}
  h_{\text{mortality}}(a)
  = - \log \Prob\{\text{Survival for $\unit[1]{y}$}\},
\end{equation}
and the survival
\begin{equation}
  \begin{split}
    S_{\text{mortality}}(a)
    =
    \begin{cases}
      0.66^{a_{\mathrm{y}}}
      & \text{if $a < \unit[1]{y}$},
      \\
      0.66 \cdot 0.79^{a_{\mathrm{y}} - 1}
      & \text{if $\unit[1]{y} \leq a < \unit[3]{y}$},
      \\
      0.66 \cdot 0.79^2 \cdot 0.88^{a_{\mathrm{y}} - 3}
      & \text{if $\unit[3]{y} \leq a < \unit[12]{y}$},
      \\
      0.66 \cdot 0.79^2 \cdot 0.88^9 \cdot 0.66^{a_{\mathrm{y}} - 12}
      & \text{if $a \geq \unit[12]{y}$},
    \end{cases}
  \end{split}
\end{equation}
with years of age $a_{\mathrm{y}} = \frac{a}{\unit[1]{y}}$.

\section{Birth}

We assumed that females reach reproductive maturity at age $\unit[4]{y}$
and that the birth hazard varies at a periodic, triangular-shaped rate in
time (\autoref{fig:birth_hazard}).
Define the year fractional part
\begin{equation}
  \{t\}_{\mathrm{y}}
    = \left\{\frac{t}{\unit[1]{y}}\right\},
\end{equation}
where $\{x\}$ is the standard fractional-part function,
and the year floor
\begin{equation}
  \lfloor t \rfloor_{\mathrm{y}}
    = \left\lfloor\frac{t}{\unit[1]{y}}\right\rfloor,
\end{equation}
where $\lfloor x \rfloor$ is the standard floor function.
The hazard is then
\begin{equation}
  h_{\text{birth}}(t, a) =
  \begin{cases}
    0 & \text{if $a < \unit[4]{y}$},
    \\
    \mu \alpha \max\big(
    1 - \beta (1 - |1 - 2 \{t - \tau\}_{\mathrm{y}}|),
    0
    \big)
    & \text{if $a \geq \unit[4]{y}$},
  \end{cases}
\end{equation}
with
\begin{equation}
  \begin{split}
    \alpha &=
    \begin{cases}
      1 + c_{\text{v}} \sqrt{3}
      & \text{if $c_{\text{v}} < \frac{1}{\sqrt{3}}$},
      \\
      \frac{3}{2} \left(1 + c_{\text{v}}^2\right)
      & \text{if $c_{\text{v}} \geq \frac{1}{\sqrt{3}}$},
    \end{cases}
    \\
    \beta &=
    \begin{cases}
      \frac{2 c_{\text{v}} \sqrt{3}}{1 + c_{\text{v}} \sqrt{3}}
      & \text{if $c_{\text{v}} < \frac{1}{\sqrt{3}}$},
      \\
      \frac{3}{4} \left(1 + c_{\text{v}}^2\right)
      & \text{if $c_{\text{v}} \geq \frac{1}{\sqrt{3}}$}.
    \end{cases}
  \end{split}
\end{equation}
The magnitude of the seasonal variation is captured by the coefficient
of variation $c_{\text{v}}$.  The time of year of the peak birth
hazard is $\tau$.  The annual mean $\mu$ was determined
so that the population has asymptotic growth rate $r = 0$.  (See
\autoref{stable_age_distribution}.)

\begin{figure}
  \centering
  \begin{sansmath}
    \input{birth_hazard.pgf}
  \end{sansmath}
  \caption{Model birth hazards for ages $\unit[4]{y}$ and older.}
  \label{fig:birth_hazard}
\end{figure}

The cumulative hazard at time $t$ given age $a_0$ at the current
time $t_0$ is
\begin{equation}
  H_{\text{birth}}(t, t_0, a_0) =
  \begin{cases}
    0 & \text{if $a_0 + t < \unit[4]{y}$},
    \\
    \mu_{\mathrm{y}} \left(H_0 + H_1  + H_2\right)
    & \text{if $a_0 + t \geq \unit[4]{y}$
      and $c_{\text{v}} < \frac{1}{\sqrt{3}}$},
    \\
    \mu_{\mathrm{y}} \left(H_0 + H_3 + H_4\right)
    & \text{if $a_0 + t \geq \unit[4]{y}$
      and $c_{\text{v}} \geq \frac{1}{\sqrt{3}}$},
  \end{cases}
\end{equation}
with
\begin{equation}
  \begin{split}
    \mu_{\mathrm{y}} &= \mu \cdot (\unit[1]{y}),
    \\
    c &= t_0 + \max(\unit[4]{y} - a_0, 0) - \tau,
    \\
    d &= t_0 + t - \tau,
    \\
    H_0 &= \lfloor d \rfloor_{\mathrm{y}} - \lfloor c \rfloor_{\mathrm{y}} - 1,
    \\
    H_1 &=
    \begin{cases}
      \frac{1}{2}
      + \alpha \left(\frac{1}{2} - \{c\}_{\mathrm{y}}\right)
      \left[1 - \beta
        + \beta \left(\frac{1}{2} - \{c\}_{\mathrm{y}}\right)\right]
      & \text{if $\{c\}_{\mathrm{y}} < \frac{1}{2}$},
      \\
      \alpha \left(1 - \{c\}_{\mathrm{y}}\right)
      \left[1 - \beta + \beta \left(1 - \{c\}_{\mathrm{y}}\right)\right]
      & \text{if $\{c\}_{\mathrm{y}} \geq \frac{1}{2}$},
    \end{cases}
    \\
    H_2 &=
    \begin{cases}
      \alpha \{d\}_{\mathrm{y}}\left(1 - \beta \{d\}_{\mathrm{y}}\right)
      & \text{if $\{d\}_{\mathrm{y}} < \frac{1}{2}$},
      \\
      \frac{1}{2}
      + \alpha \left(\{d\}_{\mathrm{y}} - \frac{1}{2}\right)
      \left[1 - \beta
        + \beta \left(\{d\}_{\mathrm{y}} - \frac{1}{2}\right)\right]
      & \text{if $\{d\}_{\mathrm{y}} \geq \frac{1}{2}$},
    \end{cases}
    \\
    H_3 &=
    \begin{cases}
      \frac{1}{2} + \alpha \beta \left(\frac{1}{2 \beta} - \{c\}_{\mathrm{y}}\right)^2
      & \text{if $\{c\}_{\mathrm{y}} < \frac{1}{2 \beta}$},
      \\
      \frac{1}{2}
      & \text{if $\frac{1}{2 \beta} \leq \{c\}_{\mathrm{y}} < 1 - \frac{1}{2 \beta}$},
      \\
      \alpha \left(1 - \{c\}_{\mathrm{y}}\right) \left[1 -
        \beta \left(1 - \{c\}_{\mathrm{y}}\right)\right]
      & \text{if $\{c\}_{\mathrm{y}} \geq 1 - \frac{1}{2 \beta}$},
    \end{cases}
    \\
    H_4 &=
    \begin{cases}
      \alpha \{d\}_{\mathrm{y}} \left[1 - \beta \{d\}_{\mathrm{y}}\right]
      & \text{if $\{d\}_{\mathrm{y}} < \frac{1}{2 \beta}$},
      \\
      \frac{1}{2}
      & \text{if $\frac{1}{2 \beta} \leq \{d\}_{\mathrm{y}} <
        1 - \frac{1}{2 \beta}$},
      \\
      \frac{1}{2}
      + \alpha \beta
      \left[\{d\}_{\mathrm{y}} - \left(1 - \frac{1}{2 \beta}\right)\right]^2
      & \text{if $\{d\}_{\mathrm{y}} \geq 1 - \frac{1}{2 \beta}$}.
    \end{cases}
  \end{split}
\end{equation}
The survival function for $t$ years given age
$a_0$ at the current time $t_0$ is then
\begin{equation}
  S_{\text{birth}}(t, t_0, a_0) = \exp\left(- H_{\text{birth}}(t, t_0, a_0)\right).
\end{equation}
The probability density function for births time $t$ later, given age
$a_0$ at $t_0$, is
\begin{align}
  f_{\text{birth}}(t, t_0, a_0) =
  h_{\text{birth}}(t_0 + t, a_0 + t)
  S_{\text{birth}}(t, t_0, a_0).
\end{align}

% Multiplying the probability density by $N(t_0, a_0)$, the density of
% females aged $a_0$ at $t_0$, and integrating over $a_0$ gives the
% expected number of births time $t$ later:
% \begin{equation}
%   b(t, t_0) = \int_0^{\infty} f_{\text{birth}}(t, t_0, a_0) N(t_0, a_0) \md a_0.
% \end{equation}
% Binning by month gives the expected number of births in month $m$:
% \begin{align}
%   g(m, t_0) &=
%   \int_0^1 b\left(\frac{m + \mu}{12}, t_0\right) \md \mu
%   & & \text{for $m \in \{0, 1, 2, \ldots\}$}.
% \end{align}

On birth, a newborn is female with probability
$p_{\text{female}} = 0.5$ or otherwise male.


\section{Waning}

The duration of maternal immunity was taken to be a standard gamma
random variable with shape $k_{\text{waning}}$
and mean $\mu_{\text{waning}}$.


\section{Infection}

The infection hazard was taken to be
\begin{equation}
  h_{\text{infection}}(t) = \beta_{\text{acute}} I(t) +
  \beta_{\text{chronic}} C(t),
\end{equation}
where $I(t)$ and $C(t)$ are the total number of infectious and
chronic-carrier buffalo in the herd at time $t$.  Over periods where
$I(t)$ and $C(t)$ are constant, the hazard is constant, which gives an
exponential random variable.


\section{Progression}

The duration of the latent period was taken to be a standard gamma
random variable with shape $k_{\text{progression}}$
and mean $\mu_{\text{progression}}$.


\section{Recovery}

The duration of infection was taken to be a standard gamma random
variable with shape $k_{\text{recovery}}$ and mean
$\mu_{\text{recovery}}$.

On recovery, buffalo become chronic carriers with probability
$p_{\text{chronic}}$ or otherwise are recovered, i.e.~fully cleared
of pathogen.


\section{Chronic recovery}

The duration of the chronic-carrier state was taken to be an
exponential random variable with mean
$\mu_{\text{chronic recovery}}$.


\begin{figure}
  \centering
  \begin{sansmath}
    \input{distributions.pgf}
  \end{sansmath}
  \caption{Hazards and survivals for the model events.}
  \label{fig:distributions}
\end{figure}


\section{Initial conditions}

A sample of size $N$ from the stable age distribution was used to
initialize the population.  (See \autoref{stable_age_distribution}.)
The sex of each buffalo was randomly selected with probability
$p_{\text{female}}$ of being female.

\textbf{TODO: HOW TO DO INITIAL CONDITIONS.}

\cite{hedger_1972}

The simulations were started at $t = 0$ with $2$ initial infections
in randomly chosen susceptible in the population.

The model was initiated at $t_0 = \unit[0.5]{y}$, i.e.~about 1 July.
The peak of the birth hazard was at $\tau = 0$, i.e.~1 January.
Plot of sensitivity to $t_0$?

\section{Stable age distribution}
\label{stable_age_distribution}

Because there is no additional mortality due to the pathogen, the mean
density of female buffaloes of age $a$ satisifies the McKendrick--von
Foerster partial differential equation (PDE)
\begin{equation}
  \label{PDE}
  \begin{split}
    \frac{\partial n}{\partial t}(t, a)
    + \frac{\partial n}{\partial a}(t, a)
    &= - h_{\text{death}}(a) n(t, a),
    \\
    n(t, 0) &=
    p_{\text{female}}
    \int_0^{+\infty} h_{\text{birth}}(t, a) n(t, a) \md a,
    \\
    n(t_0, a) &= n_0(a).
  \end{split}
\end{equation}
where $n_0(a)$ is the initial density
% Chapter VI, Section 29 on pp.~159--161 of harris_1963
% Chapter 20 on pp.~353--364 of kot_01
\autocite{harris_1963, kot_01}.

Because the birth hazard, $h_{\text{birth}}(t, a)$,
is periodic with period $T = \unit[1]{y}$,
we found the stable age distribution and the asymptotic population
growth rate using Floquet theory \autocite{parker_1992}.  Floquet
theory requires the fundamental solution $\Phi(t, a, a')$
for McKendrick--von Foerster equation \eqref{PDE}, which, for each
$a'$,
satisfies the same PDE and birth integral as $n(t, a)$,
but with an initial condition localized to age $a'$:
\begin{equation}
  \label{fundamental_PDE}
  \begin{split}
    \frac{\partial \Phi}{\partial t}(t, a, a')
    + \frac{\partial \Phi}{\partial a}(t, a, a')
    &= - h_{\text{death}}(a) \Phi(t, a, a'),
    \\
    \Phi(t, 0, a') &=
    p_{\text{female}}
    \int_0^{+\infty} h_{\text{birth}}(t, a) \Phi(t, a, a') \md a,
    \\
    \Phi(t_0, a, a') &= \delta(a - a'),
  \end{split}
\end{equation}
where $\delta(x)$ is the Dirac delta.

To solve this numerically, we used the Crank--Nicolson method on
characteristics and the composite trapezoid rule for the birth
integral \autocite{milner_1992}.  Given the time step $\Delta t$,
let $a_i = i \Delta t$
and $a'_j = j \Delta t$
for $i, j \in \{0, 1, 2, \ldots, I - 1\}$;
$t^k = t_0 + k \Delta t$
for $k \in \{0, 1, \ldots, K - 1\}$;
and $\Phi_{i, j}^k \approx \Phi(t^k, a_i, a'_j)$.
For each $j$
and each $k \geq 1$, the Crank--Nicolson method on characteristics is
\begin{equation}
  \label{CN_step}
  \frac{\Phi_{i, j}^k - \Phi_{i - 1, j}^{k - 1}}{\Delta t}
  = - h_{\text{death}}(a_{i - 1 / 2})
  \frac{\Phi_{i, j}^k + \Phi_{i - 1, j}^{k - 1}}{2},
\end{equation}
or
\begin{equation}
  \Phi_{i, j}^k
  = \frac{1 - C_{i - 1 / 2}}{1 + C_{i - 1 / 2}}
  \Phi_{i - 1, j}^{k - 1},
\end{equation}
with
\begin{equation}
  C_{i - 1 / 2}
  = \frac{h_{\text{death}}(a_{i - 1 / 2}) \Delta t}{2},
\end{equation}
for $i \in \{1, 2, \ldots, I - 2\}$.  For $i = I - 1$,
a term was added to prevent buffaloes from aging out of this
last age group:
\begin{equation}
  \Phi_{I - 1, j}^k
  = \frac{1 - C_{I - 3 / 2}}{1 + C_{I - 3 / 2}}
  \Phi_{I - 2, j}^{k - 1}
  + \frac{1 - C_{I - 1}}{1 + C_{I - 1}}
  \Phi_{I - 1, j}^{k - 1},
\end{equation}
with
\begin{equation}
  C_{I - 1}
  = \frac{h_{\text{death}}(a_{I - 1}) \Delta t}{2}.
\end{equation}
For $i = 0$, the birth integral is given by the composite trapezoid rule,
\begin{equation}
  \label{birth_step}
  \Phi_{0, j}^k =
  \frac{p_{\text{female}} \Delta t}{2}
  \sum_{i = 1}^{I - 1}
  \left[h_{\text{birth}}(t^k, a_i) \Phi_{i, j}^k +
    h_{\text{birth}}(t^k, a_{i - 1}) \Phi_{i - 1, j}^k\right].
\end{equation}
The initial condition is
\begin{equation}
  \Phi_{i, j}^0 =
  \begin{cases}
    1 & \text{if $i = j$}, \\
    0 & \text{otherwise}.
  \end{cases}
\end{equation}
Considering $\mat{\Phi}^k = [\Phi_{i, j}^k]$ as a matrix that
evolves in time, the method is easily implemented with matrix algebra:
the Crank--Nicolson step \eqref{CN_step} is
\begin{equation}
  \mat{\Phi}^k = \mat{M} \mat{\Phi}^{k - 1},
\end{equation}
with the matrix $\mat{M} = [M_{i, j}]$ where
\begin{equation}
  M_{i, j} =
  \begin{cases}
    \frac{1 - C_{i - 1 / 2}}{1 + C_{i - 1 / 2}}
    & \text{if $i = j + 1$}, \\
    \frac{1 - C_{I - 1}}{1 + C_{I - 1}} & \text{if $i = j = I - 1$}, \\
    0 & \text{otherwise}.
  \end{cases}
\end{equation}
Because the birth hazard can be decomposed into the product of a
time-varying part and an age-varying part,
\begin{equation}
  h_{\text{birth}}(t^k, a_i)
  = \hat{h}_{\text{birth}}(t^k) \bar{h}_{\text{birth}}(a_i),
\end{equation}
the birth integral \eqref{birth_step} is then
\begin{equation}
  \mat{\Phi}_0^k = \hat{h}_{\text{birth}}(t^k) \vec{v} \mat{\Phi}^k,
\end{equation}
with the vector $\vec{v} = [v_i]$ for
\begin{equation}
  v_i =
  \begin{cases}
    \frac{1}{2} p_{\text{female}} \bar{h}_{\text{birth}}(a_i) \Delta t
    & \text{if $i = 0$ or $i = I - 1$}, \\
    p_{\text{female}} \bar{h}_{\text{birth}}(a_i) \Delta t
    & \text{otherwise}.
  \end{cases}
\end{equation}
The initial condition is
\begin{equation}
  \mat{\Phi}^0 = \mat{I},
\end{equation}
where $\mat{I}$ is the $I \times I$ identity matrix.

Using this numerical scheme, we solved for the monodromy matrix, the
fundamental solution after one period:
\begin{equation}
  \mat{\Psi} = [\Psi_{i, j}] \approx [\Phi(T, a_i, a'_j)].
\end{equation}
The monodromy matrix projects the population forward at by one period,
\begin{equation}
  \vec{n}(t_0 + T) = \mat{\Psi} \vec{n}(t_0),
\end{equation}
where $\vec{n}(t) = [n(t, a_i)]$.
Using the monodromy operator to repeatedly project the population
forward gives
\begin{equation}
  \vec{n}(t_0 + K T)
  = \mat{\Psi}^K \vec{n}(t_0)
  \to \rho_0^K \vec{w}_0
  = \me^{r K T} \vec{w}_0
\end{equation}
as $K \to \infty$,
where $\rho_0$ is the dominant eigenvalue, i.e.~the eigenvalue with
largest magnitude, of $\mat{\Psi}$;
the corresponding right eigenvector $\vec{w}_0$ is the stable age
distribution; and
\begin{equation}
  r = \frac{1}{T} \log \rho_0
\end{equation}
is the asymptotic population growth rate.

We numerically computed the population growth rate and stable age
distribution by using time step $\Delta t = \unit[0.01]{y}$,
maximum age $a_{\text{max}} = \unit[35]{y}$
(probability of survival
$S_{\text{mortality}}(\unit[35]{y}) \approx 10^{-5}$),
finding the monodromy matrix by numerically solving
\eqref{fundamental_PDE} from $t = t_0$ to $T$, and
then finding its dominant eigenvalue and corresponding eigenvector.
We then used a root-finding algorithm to find the value of the mean
birth hazard $\mu \approx \unit[0.9379]{y^{-1}}$ that gave growth rate
$r = 0$ (\autoref{fig:stable_age_distribution}). Halving the time
step to $\Delta t = \unit[0.005]{y}$ gave a relative error for $\mu$
of $7 \times 10^{-4}$ and doubling the maximum age to $a_{\text{max}}
= \unit[70]{y}$ gave a relative error of $5 \times 10^{-6}$.

\begin{figure}
  \centering
  \begin{sansmath}
    \input{stable_age_distribution.pgf}
  \end{sansmath}
  \caption{The stable age distribution in the buffalo population
    at $t_0 = \unit[0.5]{y}$, i.e.~about 1 July. The peak in the birth
    hazard is on 1 January ($\tau = 0$).}
  \label{fig:stable_age_distribution}
\end{figure}

\textbf{TODO: HOW TO DO SAMPLING}


\printbibliography


\newpage
\appendix
\section*{Supplemental Results Figures}
% These figure numbers will start after the previous appendix.
% I.e. if the previous was Appendix S3, these will be Figure S4,
% Figure S5, ...
\setcounter{figure}{\value{chapter}}
\renewcommand{\thefigure}{\appendixlabel\arabic{figure}}
\renewcommand{\HyperDestNameFilter}[1]{chapter-#1}


\begin{figure}
  \centering
  \includegraphics{../population_sizes}
  \caption{The sensitivity of extinction time of FMDV to buffalo
    population size.
    For each model and each SAT, the model was simulated for 1000 runs
    at population sizes 100, 200, 300, 400, 500, 600, 700, 800, 900,
    1000 (baseline, dotted vertical lines), 2000, 3000, 4000, and
    5000.
    The other parameters were fixed at their baseline values.
    The top and middle rows of graphs show the distribution of
    FMDV extinction times for the model with only acute transmission
    and the model with both acute and chronic transmission,
    respectively.
    The bottom row shows the proportion of simulations where FMDV
    persisted in the buffalo population for the whole simulated
    10-year period for the model with both acute and chronic
    transmission.}
\end{figure}

\begin{figure}
  \centering
  \includegraphics{../birth_seasonality}
  \caption{The sensitivity of extinction time to birth seasonality.
    For each model and each SAT, the model was simulated for
    1000 runs at 0, 0.5, 1 (baseline, dotted vertical lines), 1.5, and
    2 times the baseline birth seasonal coefficient of variation of
    0.613.
    The other parameters were fixed at their baseline values.
    The top and middle rows of graphs show the distribution of
    FMDV extinction times for the model with only acute transmission
    and the model with both acute and chronic transmission,
    respectively.
    The bottom row shows the proportion of simulations where FMDV
    persisted in the buffalo population for the whole simulated
    10-year period with both acute and chronic transmission.}
\end{figure}

\begin{figure}
  \centering
  \includegraphics{../samples_sensitivity_acute}
  \caption{The sensitivity of FMDV extinction time to model
    parameters, for the model with only acute transmission.
    The sensitivity is measured by the partial rank correlation
    coefficient (PRCC) \autocite{blower_1994}.
    The model was simulated with each of 20,000 samples from the
    posterior distributions of the parameters
    (Fig.~2; Tables S2, S3).}
\end{figure}


\end{document}
